%=======================02-713 LaTeX template, following the 15-210 template==================
%
% You don't need to use LaTeX or this template, but you must turn your homework in as
% a typeset PDF somehow.
%
% How to use:
%    1. Update your information in section "A" below
%    2. Write your answers in section "B" below. Precede answers for all 
%       parts of a question with the command "\question{n}{desc}" where n is
%       the question number and "desc" is a short, one-line description of 
%       the problem. There is no need to restate the problem.
%    3. If a question has multiple parts, precede the answer to part x with the
%       command "\part{x}".
%    4. If a problem asks you to design an algorithm, use the commands
%       \algorithm, \correctness, \runtime to precede your discussion of the 
%       description of the algorithm, its correctness, and its running time, respectively.
%    5. You can include graphics by using the command \includegraphics{FILENAME}
%
\documentclass[11pt]{article}

\usepackage{amsmath,amssymb,amsthm}
\usepackage{graphicx}
\usepackage[margin=1in]{geometry}
\usepackage{fancyhdr}
\usepackage{comment}
\usepackage{tikz}
\usetikzlibrary{arrows}
\usepackage{caption}
\usepackage{subcaption}


\setlength{\parindent}{0pt}
\setlength{\parskip}{5pt plus 1pt}
\setlength{\headheight}{13.6pt}
\newcommand\question[2]{\vspace{.25in}\hrule\textbf{#1: #2}\vspace{.5em}\hrule\vspace{.10in}}
\renewcommand\part[1]{\vspace{.10in}\textbf{(#1)}}
\newcommand\algorithm{\vspace{.10in}\textbf{Algorithm: }}
\newcommand\correctness{\vspace{.10in}\textbf{Correctness: }}
\newcommand\runtime{\vspace{.10in}\textbf{Running time: }}
\pagestyle{fancyplain}
\lhead{\textbf{\NAME\ \ANDREWID}}
\chead{\textbf{Assignment\HWNUM}}
\rhead{CS 440, Winter 2018}
\begin{document}\raggedright
%Section A==============Change the values below to match your information==================
\newcommand\NAME{Oregon State University}  % your name
\newcommand\ANDREWID{}     % your andrew id
\newcommand\HWNUM{4}              % the homework number
%Section B==============Put your answers to the questions below here=======================

% no need to restate the problem --- the graders know which problem is which,
% but replacing "The First Problem" with a short phrase will help you remember
% which problem this is when you read over your homeworks to study.

The assignment is to be turned in before Midnight (by 11:59pm) on February 8 , 2018. 
You should turn in the solutions to this assignment as a pdf file through the TEACH website.
The solutions should be produced using editing software programs, such as LaTeX or Word, otherwise they will not be graded.


\question{1}{Query optimization  (2 points)}
Consider the following relations:


Product (name, production-year, rating, company-name)\\
Company (name, state, employee-num)

Assume each product is produced by just one company, whose name is mentioned in the 
{\it company-name} attribute of the {\it Product} relation. 
Attributes {\it name} are the primary key for both relations {\it Product} and {\it Company}.
Attribute {\it rating} shows how popular a product is and
its values are between 1-5. 


The following query returns the products with rating of 5 that are produced after 2000 and the states of their companies.\\

\begin{verbatim}
SELECT p.name, c.state
FROM Product p, Company c
WHERE p.company-name = c.name and p.production-year > 2000 and p.rating = 5
\end{verbatim}

Suggest an optimized logical query plan for the above query.

\paragraph{Solution:} \hfill \break
For the above query the logical plan would be to first select products who's rating is 5. Given that there are only five options for this value it is likely that this at least reduces the number of products by 1/5. While doing this you would only select the company name and product name for each tuple who's rating was 5. Next you would select the items from the resulting list that were made after 2000. Next you would select Company name from C. Finally you would join the results on the company name.\\



\question{2}{Query optimization  (2 point)}
For the four base relations in the following table, 
find the best join order according to 
the dynamic programming algorithm used in System-R. 
You should give the dynamic programming table entries 
for evaluate the join orders.
Suppose that we are only interested in left-deep join trees and 
join trees without Cartesian products. 
Note that you must use the Selinger-style formulas to compute the size of each join output.
The database system uses hash join to compute every join.
Assume that there is enough main memory to perform the hash join for every pairs of relations.
Each block contains at most 5 tuples of a base or joint relation.
If relations are joined on multiple attributes, you can compute the selectivity factor of the full join by multiplying the 
selectivity factors of joining on each attribute.

		
\begin{tabular}{ |l| l | l|l|}
\hline
  R(A,B,C) & S(B,C) & W(B,D) & U(A,D) \\\hline
  T(R)=4000 & T(S)=3000& T(W)=2000 & T(U)=1000 \\\hline
  V(R,A) =100 &  & & V(U,A) =100\\
  V(R,B) =200 & V(S,B) =100& V(W,B) =100& \\
  V(R,C) =100 & V(S,C) = 300&&\\
  & & V(W,D) =50 & V(U,D) =100\\\hline
\end{tabular}





\question{3}{Concurrency control (3 points)}
Consider the following classes of schedules: serializable and 2PL. 
For each of the following schedules, state which of the preceding classes it belongs to. 
If you cannot decide whether a schedule belongs to a certain class based on the listed actions, explain briefly your reasons.

The actions are listed in the order they are scheduled and prefixed with the transaction name. 
If a commit or abort is not shown, the schedule is incomplete; assume that abort or commit must follow all the listed actions.


\begin{enumerate}
\item T1:R(X), T2:R(Y), T3:W(X), T2:R(X), T1:R(Y)
\item T1:R(X), T1:R(Y), T1:W(X), T2:R(Y), T3:W(Y), T1:W(X), T2:R(Y)
\end{enumerate}



\end{document}
