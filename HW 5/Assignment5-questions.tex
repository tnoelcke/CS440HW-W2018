%=======================02-713 LaTeX template, following the 15-210 template==================
%
% You don't need to use LaTeX or this template, but you must turn your homework in as
% a typeset PDF somehow.
%
% How to use:
%    1. Update your information in section "A" below
%    2. Write your answers in section "B" below. Precede answers for all 
%       parts of a question with the command "\question{n}{desc}" where n is
%       the question number and "desc" is a short, one-line description of 
%       the problem. There is no need to restate the problem.
%    3. If a question has multiple parts, precede the answer to part x with the
%       command "\part{x}".
%    4. If a problem asks you to design an algorithm, use the commands
%       \algorithm, \correctness, \runtime to precede your discussion of the 
%       description of the algorithm, its correctness, and its running time, respectively.
%    5. You can include graphics by using the command \includegraphics{FILENAME}
%
\documentclass[11pt]{article}

\usepackage{amsmath,amssymb,amsthm}
\usepackage{graphicx}
\usepackage[margin=1in]{geometry}
\usepackage{fancyhdr}
\usepackage{comment}
\usepackage{tikz}
\usetikzlibrary{arrows}
\usepackage{caption}
\usepackage{subcaption}


\setlength{\parindent}{0pt}
\setlength{\parskip}{5pt plus 1pt}
\setlength{\headheight}{13.6pt}
\newcommand\question[2]{\vspace{.25in}\hrule\textbf{#1: #2}\vspace{.5em}\hrule\vspace{.10in}}
\renewcommand\part[1]{\vspace{.10in}\textbf{(#1)}}
\newcommand\algorithm{\vspace{.10in}\textbf{Algorithm: }}
\newcommand\correctness{\vspace{.10in}\textbf{Correctness: }}
\newcommand\runtime{\vspace{.10in}\textbf{Running time: }}
\pagestyle{fancyplain}
\lhead{\textbf{\NAME\ \ANDREWID}}
\chead{\textbf{Assignment\HWNUM}}
\rhead{CS 440, Winter 2018}
\begin{document}\raggedright
%Section A==============Change the values below to match your information==================
\newcommand\NAME{Oregon State University}  % your name
\newcommand\ANDREWID{}     % your andrew id
\newcommand\HWNUM{5}              % the homework number
%Section B==============Put your answers to the questions below here=======================

% no need to restate the problem --- the graders know which problem is which,
% but replacing "The First Problem" with a short phrase will help you remember
% which problem this is when you read over your homeworks to study.

The assignment is to be turned in before noon (by 11:59 am) on February 13 , 2018. 
You should turn in the solutions to this assignment as a pdf file through the TEACH website.
The solutions should be produced using editing software programs, such as LaTeX or Word, otherwise they will not be graded.


\question{1}{Multi-granularity Locking (1 point)}
Consider a database DB with relations R1 and R2. 
The relation R1 contains tuples t1 and t2 and the relation R2 contains
tuples t3, t4, and t5. Assume that the database DB, relations, and tuples form a hierarchy of lockable database elements.
Explain the sequence of lock requests and the response of the locking scheduler to the following schedule. 
You may assume all lock requests occur just before they are needed, and all unlocks occur at the end of the transaction.
\vspace{0.5em}

\begin{itemize}
\item T1:R(t1), T2:W(t2), T2:R(t3), T1:W(t4)
\end{itemize}

\question{2}{Degrees of Consistency (1 point)}
Consider the schedule shown at Table~\ref{schedule}.

\begin{table}[h!]
\begin{tabular}{ l |l |l |l |}
   &  T1 & T2 \\
  0 & start &   \\
  1 & read X &  \\
  2 & write X &   \\
  3 &  &  start \\
  4 &  &  read X \\
  5 &  &  write X \\
  6 &  &  Commit \\
  7 & read Y&  \\
  8 & write Y&  \\
  9 & Commit&  \\
\end{tabular}
\caption{Transaction schedule}
\label{schedule} 
\end{table}
\vspace{-1em} 

What are the maximum degrees of consistency for T1 and T2
in this schedule? You must find the maximum degrees of consistency for T1 and T2 that makes this schedule possible.
The degree of consistency for T1 may be different from the degree of consistency of T2.

\end{document}
