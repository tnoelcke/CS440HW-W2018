%=======================02-713 LaTeX template, following the 15-210 template==================
%
% You don't need to use LaTeX or this template, but you must turn your homework in as
% a typeset PDF somehow.
%
% How to use:
%    1. Update your information in section "A" below
%    2. Write your answers in section "B" below. Precede answers for all 
%       parts of a question with the command "\question{n}{desc}" where n is
%       the question number and "desc" is a short, one-line description of 
%       the problem. There is no need to restate the problem.
%    3. If a question has multiple parts, precede the answer to part x with the
%       command "\part{x}".
%    4. If a problem asks you to design an algorithm, use the commands
%       \algorithm, \correctness, \runtime to precede your discussion of the 
%       description of the algorithm, its correctness, and its running time, respectively.
%    5. You can include graphics by using the command \includegraphics{FILENAME}
%
\documentclass[11pt]{article}
\usepackage{amsmath,amssymb,amsthm}
\usepackage{graphicx}
\usepackage[margin=1in]{geometry}
\usepackage{fancyhdr}
\usepackage{comment}
\usepackage{listings}
\setlength{\parindent}{0pt}
\setlength{\parskip}{5pt plus 1pt}
\setlength{\headheight}{13.6pt}
\newcommand\question[2]{\vspace{.25in}\hrule\textbf{#1: #2}\vspace{.5em}\hrule\vspace{.10in}}
\renewcommand\part[1]{\vspace{.10in}\textbf{(#1)}}
\newcommand\algorithm{\vspace{.10in}\textbf{Algorithm: }}
\newcommand\correctness{\vspace{.10in}\textbf{Correctness: }}
\newcommand\runtime{\vspace{.10in}\textbf{Running time: }}
\pagestyle{fancyplain}
\lhead{\textbf{\NAME\ \ANDREWID}}
\chead{\textbf{Assignment \HWNUM}}
\rhead{CS 440, Winster 2018}
\begin{document}\raggedright
%Section A==============Change the values below to match your information==================
\newcommand\NAME{Thomas Noelcke}  % your name
\newcommand\ANDREWID{noelcket}     % your andrew id
\newcommand\HWNUM{1}              % the homework number
%Section B==============Put your answers to the questions below here=======================

% no need to restate the problem --- the graders know which problem is which,
% but replacing "The First Problem" with a short phrase will help you remember
% which problem this is when you read over your homeworks to study.
The assignment is to be turned in before Midnight (by 11:59pm) on January 18th. You should turn in the solutions to this assignment as a PDF file through the TEACH website. The solutions should be produced using editing software programs, such as LaTeX or Word, otherwise they will not be graded.

\question{1}{Relational Model and SQL (8 points)} 
Consider the following relational  schema:\\

Emp(\underline{{\it eid}:integer}, {\it ename}:string, {\it age}:integer, {\it salary}:real)\\ 
Works(\underline{{\it eid}:integer}, \underline{{\it did}:integer}, {\it pc\_time}:integer)\\
Dept(\underline{{\it did}:integer}, {\it dname}:string, {\it budget}:real, {\it managerid}:integer)\\
\vspace{10pt}
 
The underlined attributes are keys for their relations.
Note that a manager is an employee as well and their manager id and employee id are the same. 
An employee can work in more than one department. 
The pct\_time field of the Works relation shows the percentage of time that a given employee works in a given department. 
Write the following queries in SQL.


\part{a} Print the {\it did} and {\it dname} of the departments with at least one full-time (100\%) employee. (1 point)
\part{solution}
\begin{lstlisting}[language=SQL]
	SELECT did, dname 
	 FROM dept
	 WHERE dept.did
	  IN (SELECT did
	      FROM works
	      WHERE works.pct_time > 99);
\end{lstlisting}

\part{b} Print the names and ages of each employee who works in both the "Hardware" department and the "Software" department. (1 point)
\part{solution}
\begin{lstlisting}[language=SQL]
	 SELECT ename, age FROM emp WHERE emp.eid IN 
	(SELECT eid FROM
		(SELECT eid 
			FROM works as w
				WHERE w.did = 
					(SELECT did 
					 FROM dept AS d 
					 WHERE dname = 'Software')) AS es
				WHERE eid IN 
					(SELECT eid
					 FROM works as w
					 WHERE w.did = 
					 (SELECT did 
					 FROM dept AS d 
					 WHERE dname = 'Hardware')));
\end{lstlisting}


\part{c} Print the name of each employee whose salary does {\it not} exceed the budget of any department that he or she works in. (2 point)
\part{solution}

\begin{lstlisting}[language=SQL]
	SELECT emp.ename 
  FROM emp
  WHERE emp.eid NOT IN
  (SELECT es.eid 
    FROM (SELECT emp.eid, emp.salary, works.did
    FROM emp, works
    WHERE emp.eid = works.eid) AS es
    WHERE es.salary > (SELECT budget from dept WHERE dept.did = es.did));
\end{lstlisting}

\part{d} If a manager manages more than one department, he or she controls the sum of all the budgets for those departments. Find the managerids of managers who control more than \$5 million. (2 points)
\part{solution}

\begin{lstlisting}[language=SQL]
	SELECT managerid FROM dept
		GROUP BY dept.managerid
		HAVING sum(budget) > 5000000 
\end{lstlisting}


\part{e} For each department with more than 4 full-time-equivalent employees (i.e., where the part-time and full-time employees add up to at least that many full-time employees), print the did together with the number of employees that work in that department. (2 points)
\part{solution}
\begin{lstlisting}[language=SQL]
	SELECT did, COUNT(eid)
		FROM WORKS
		GROUP BY did
		HAVING SUM(pct_time) > 400;
\end{lstlisting}
\end{document}
